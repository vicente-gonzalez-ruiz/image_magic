\title{\href{http://www.imagemagick.org/}{ImageMagick}}

\maketitle
\tableofcontents

\chapter{Introduction}
%{{{

\section{About \href{http://www.imagemagick.org/Usage/}{ImageMagick}}
%{{{

\begin{itemize}
\item ImageMagick® is a software suite to create, edit, compose, or
  convert bitmap (raster) and vector images.
\item It runs on Linux, Windows, Mac Os X, iOS, Android OS, and
  others.
\item The functionality of ImageMagick is typically utilized from the
  command line, although you can use the features from programs
  written in your favorite language (Perl, C, Ruby and PHP).
\item ImageMagick is free software distributed under
  \href{http://www.apache.org/licenses/LICENSE-2.0.html}{The Apache
    2.0 license}.
\item There is a
  \href{http://www.imagemagick.org/discourse-server/}{forum} where you
  can answer your questions.
\end{itemize}

%}}}

\begin{equation}
y=x^2
\end{equation}

\section{Installed version}
%{{{

\begin{lstlisting}
convert -version
\end{lstlisting}

%}}}

\section{\href{http://www.imagemagick.org/script/formats.php}{Listing the supported image formats}}
%{{{

\begin{lstlisting}
identify -list format
\end{lstlisting}

%}}}

\section{Delegates}
%{{{

\begin{itemize}
\item Depending on how ImageMagick was configured, there is a
  collection of delegate programs (for example, to create a MPEG
  file ImageMagick uses FFMPEG).
\begin{lstlisting}
convert -list delegate
\end{lstlisting}

\end{itemize}

%}}}

\section{(Local) Help}
%{{{

\begin{itemize}
\item All ImageMagick commands has man page:
\begin{lstlisting}
man imagemagick
\end{lstlisting}
\item Each ImageMagick command has a buit-in help:
\begin{lstlisting}
convert -help
\end{lstlisting}
\end{itemize}

%}}}

\section{Debugging}
%{{{

\begin{itemize}
\item Standard debug:
\begin{lstlisting}
wget http://www.hpca.ual.es/~vruiz/images/lena.png .
convert lena.png lena.jpg
\end{lstlisting}
\item No debug:
\begin{lstlisting}
convert +debug lena.png lena.jpg
\end{lstlisting}
\item Debug:
\begin{lstlisting}
convert -debug lena.png lena.jpg
\end{lstlisting}
\item More debug:
\begin{lstlisting}
convert -verbose lena.png lena.jpg
\end{lstlisting}
\item Progress information:
\begin{lstlisting}
convert -monitor lena.png lena.png
\end{lstlisting}

\end{itemize}

%}}}

\section{Commands}
%{{{

\begin{enumerate}
\item \href{http://www.imagemagick.org/script/animate.php}{\tt
    animage}: plays an image sequence.
\item \href{http://www.imagemagick.org/script/compare.php}{\tt
    compare}: compares images.
\item \href{http://www.imagemagick.org/script/composite.php}{\tt
    composite}: overlaps images.
\item \href{http://www.imagemagick.org/script/conjure.php}{\tt
    conjure}: performs custom image processing tasks from a script
  written in the Magick Scripting Language (MSL).
\item \href{http://www.imagemagick.org/script/convert.php}{\tt
    convert}: resizes, blurs, crops, despeckles, dithers, draws on,
  flips, joins, re-samples, and much more.
\item \href{http://www.imagemagick.org/script/display.php}{\tt
    display}: shows an image.
\item \href{http://www.imagemagick.org/script/identify.php}{\tt
    identify}: describes the format and characteristics of one or more
  images.
\item \href{http://www.imagemagick.org/script/import.php}{\tt import}:
  captures images from the screen.
\item \href{http://www.imagemagick.org/script/mogrify.php}{\tt mogrify}:
  identical to convert, but overwrites the original image file.
\item \href{http://www.imagemagick.org/script/montage.php}{\tt
    montage}: composites images by combining others.
\item \href{http://www.imagemagick.org/script/stream.php}{\tt stream}:
  extracts regions directly from disk to disk (without loading the
  image in the memory).
\end{enumerate}

%}}}

%}}}

\chapter{Displaying}
%{{{

\section{Using \href{http://www.imagemagick.org/script/identify.php}{\texttt{identify}}}
%{{{

\begin{itemize}
\item Describe the format and characteristics of one or more image files.
\begin{lstlisting}
# Basic information about the built-in "rose" image:
identify rose:

# Idem for lena.png:
identify lena.png

# Extremely basic:
identify -ping rose:

# Get extra information:
identify -verbose rose:
\end{lstlisting}

\end{itemize}

%}}}

\section{Using \href{http://www.imagemagick.org/script/display.php}{\texttt{display}}}
%{{{

\begin{itemize}
\item Display an image or image sequence on any X server.
\begin{lstlisting}
# Create an image with the inline image "rose".
convert rose: rose.png

# Display de image using an independent window.
display rose.png &

# Using the root window.
display -size 1280x1024 -window root rose.png &

# Built a preview in a visual image directory.
wget http://www.hpca.ual.es/~vruiz/videos/tempete_352x288x30x420x260.avi
convert tempete_352x288x30x420x260.avi tempete%03d.png
display 'vid:*.png' &

# Using pipes:
convert logo: gif:- | display - &
#                 ^ ^         ^
#                 | |         |
#            stdout pipe    stdin
\end{lstlisting}

\end{itemize}

%}}}

\section{Using \href{http://www.imagemagick.org/script/animate.php}{\texttt{animate}}}
%{{{

\begin{itemize}
\item Animate an image sequence on any X server.
\begin{lstlisting}
# Displays a sequence of independent JPEG images.
animate *.jpg

# Displays an animated GIF.
animate tempete_animated.gif &

# or ...
firefox tempete_animated.gif &
\end{lstlisting}
\verb|# See Section| \ref{animated} \verb|to get tempete_animated.gif.|

\end{itemize}

%}}}

\section{Using \href{http://en.wikipedia.org/wiki/File_(command)}{\texttt{file}}}

%{{{

\begin{lstlisting}
# Create the image "rose.png":
convert rose: rose.png

# Get information:
file rose.png
\end{lstlisting}

%}}}

%}}}

\chapter{Capturing}
%{{{

\begin{itemize}
\item Save any visible window on an X server and outputs it as an
  image file. You can capture a single window, the entire screen, or
  any rectangular portion of the screen.
\end{itemize}

\begin{enumerate}

\item Capture a window:

\begin{lstlisting}
import window.png
\end{lstlisting}

\item Capture a region:

\begin{lstlisting}
import -frame region.png
\end{lstlisting}

\item Capture the entire screen:

\begin{lstlisting}
import -window root screen.png
\end{lstlisting}

\end{enumerate}

%}}}

\chapter{Creating \& converting}
%{{{

\begin{itemize}
\item Convert between image formats as well as resize an image, blur,
  crop, despeckle, dither, draw on, flip, join, re-sample, and much
  more.
\end{itemize}

\section{Using built-in images}
%{{{

\begin{lstlisting}
# Create a image with a checkerboard.
convert -size 640x480 pattern:checkerboard checkerboard.png

# Show percent completion of a task as a shaded cylinder.
convert -size 320x90 canvas:none -stroke snow4 -size 1x90 -tile gradient:white-snow4 \
  -draw 'roundrectangle 16, 5, 304, 85 20,40' +tile -fill snow \
  -draw 'roundrectangle 264, 5, 304, 85  20,40' -tile gradient:chartreuse-green \
  -draw 'roundrectangle 16,  5, 180, 85  20,40' -tile gradient:chartreuse1-chartreuse3 \
  -draw 'roundrectangle 140, 5, 180, 85  20,40' +tile -fill none \
  -draw 'roundrectangle 264, 5, 304, 85 20,40' -strokewidth 2 \
  -draw 'roundrectangle 16, 5, 304, 85 20,40' \( +clone -background snow4 \
  -shadow 80x3+3+3 \) +swap -background none -layers merge \( +size -font Helvetica \
  -pointsize 90 -strokewidth 1 -fill red label:'50 %' -trim +repage \( +clone \
  -background firebrick3 -shadow 80x3+3+3 \) +swap -background none -layers merge \) \
  -insert 0 -gravity center -append -background white -gravity center -extent 320x200 \
  png:- | display -
\end{lstlisting}

%}}}

\section{Creating animated images}
%{{{

\label{animated}

\begin{enumerate}

\item Using a \emph{globbing} parameter:
%{{{

\begin{lstlisting}
# Create a sequence of PNG images.
wget http://www.hpca.ual.es/~vruiz/videos/tempete_352x288x30x420x260.avi
convert tempete_352x288x30x420x260.avi tempete%03d.png

# Create a GIF animation with a sequence of PNG images.
convert -loop 0 -delay 100 tempete*.png tempete_animated.gif
#            ^          ^         ^
#            |          |         |
#            |          | filename globbing
#            |        delay between images
#         Loop forever

firefox tempete_animated.gif &
\end{lstlisting}

%}}}

\item Using a file with the names of the still images:
%{{{

\begin{lstlisting}
cat << EOF > example_using_references.sh
#!/bin/bash

set -x

# Empty file.
echo -n "" > references.txt

# Create the "references.txt" file.
for (( i=0; i<100; i++ ))
do
  index_i=\`printf "%02d" \$i\` # Be carefully with the '\`'!
  filename=checker-\$index_i.png
  convert -size 256x256 pattern:checkerboard'['256x256+\$i+\$i']' \$filename
  echo \$filename >> references.txt
done

# Create the movie using the references file.
convert @references.txt movie.gif

set +x
EOF

chmod +x example_using_references.sh
./example_using_references.sh
\end{lstlisting}

\end{enumerate}

%}}}

%}}}

\section{Extracting images from an animated image or video}
%{{{

\begin{lstlisting}
convert tempete_352x288x30x420x260.avi tempete.gif

# Extracting the first image of an animated GIF.
convert 'tempete.gif[0]' tempete.png
display tempete.png

# Extracting a range of images.
convert 'tempete.gif[0-3]' tempete.mng
display tempete.mng

# Extracting images out of order.
convert 'tempete.gif[3,2,4]' tempete.mng
display tempete.mng
\end{lstlisting}

%}}}

\section{Disolving}
%{{{

\begin{lstlisting}
composite -disolve 50% source1.jpg source2.png creation.tif
\end{lstlisting}

%}}}

\section{Morphing}
%{{{

\begin{lstlisting}
wget http://assets.worldwildlife.org/photos/2090/images/hero_full/Sumatran-Tiger-Hero.jpg?1345559303
wget http://4.bp.blogspot.com/-OHVlacYzSlU/TZBinlfNXzI/AAAAAAAAEFI/tqaeTB2wfTA/s1600/Clara%20Lago%20Pictures.jpg
convert -morph 50 Sumatran-Tiger-Hero.jpg\?1345559303 Clara\ Lago\ Pictures.jpg output.mng
\end{lstlisting}

%}}}

\section{Watermarking}
%{{{

\begin{enumerate}

\item Visible:

\begin{lstlisting}
# No offset:
composite -watermark 50% mark.png foreground.jpg output.png

# Centering the mark:
composite -watermarck 40% -gravity center marck.jpg foreground.jpg output.jpg
\end{lstlisting}

\item Invisible:

\begin{lstlisting}
# Hide ``hidden_mark.png'' inside of ``foreground.jpg'':
composite -stegano 101 hidden_mark.png foreground.jpg output.jpg
#                   ^
#                   |
#       offset of the hidden mark

# Extract the hidden image:
display 200x100+101 stegano:output.png
#          ^     ^
#          |     |
#          | offset of the hidden mark
#  size of the hidden marck
\end{lstlisting}

\end{enumerate}

%}}}

\section{Removing and putting on headers}
%{{{

\begin{lstlisting}
file tempete000.png
# tempete000.png: PNG image data, 352 x 288, 8-bit/color RGB, non-interlaced
\end{lstlisting}
\begin{lstlisting}
convert tempete000.png rgb:tempete_352x288_8bpp_rgb.raw
#                      ^^^
#    We specify an explicit output image format.
\end{lstlisting}
\noindent~\verb|# See Section| \texttt{\ref{animated}} \verb|to create tempete000.png.|
\begin{lstlisting}
convert -size 352x288 -depth 8 rgb:tempete.rgb tempete.gif
#       ^^^^^^^^^^^^^^^^^^^^^^^^^^^

# "raw_image" is an input collection of 640x480 RGB samples without any header:
convert -size 640x480 -depth 8 rgb:raw_image image.png
#                              ^^^
\end{lstlisting}

%}}}

%}}}

\chapter{Comparing}
%{{{

\section{Compare two images}

\begin{lstlisting}
compare image1.png image2.png difference.png 
\end{lstlisting}

%}}}

\chapter{Composing}
%{{{

\section{Overlaping images}
%{{{

\begin{lstlisting}
composite first.png second.png composite.png
\end{lstlisting}

%}}}

\section{Montaging}
%{{{

\begin{itemize}
\item  Create a composite image by combining several separate images. The
images are tiled on the composite image optionally adorned with a
border, frame, image name, and more.
\begin{lstlisting}
montage -label %f -frame 10 -shadow -tile 4 *.jpg output.jpg
#              ^         ^   ^            ^
#              |         |   |            |
#              |         |   |  number of images per row
#              |         | create a shadow for each image
#              |     use a frame os 10 pixels
#            filename
\end{lstlisting}

\end{itemize}

%}}}

%}}}

%\chapter{Scripting ({\tt conjure})}
%{{{

%\section{Conjure}
%Interpret and execute scripts written in the Magick Scripting Language
%(MSL).

%}}}

\chapter{Processing}
%{{{

\section{Reducing the number of colors}
%{{{

\begin{lstlisting}
convert -colors 256 input.tif output.png
# or
convert -depth 8 input.tif output.png
\end{lstlisting}

%}}}

\section{Chroma subsampling}
%{{{

\begin{lstlisting}
# Selects YUV 4:2:2 chroma subsampling:
convert -sampling-factor 2x1 input.png output.png
\end{lstlisting}

%}}}

\section{Changing the quality}
%{{{

\begin{enumerate}
\item In JPEG:
\begin{lstlisting}
convert -quality 15% input.png output.jpg
\end{lstlisting}

\item In JPEG2000:
\begin{lstlisting}
convert -quality 15% input.png -define "jp2:rate=1.0" output.jp2
\end{lstlisting}

\end{enumerate}
%}}}

\section{Gamma correction}
%{{{

\begin{lstlisting}
conver -gamma 1.5,1.6,0.9 input.png output.png
\end{lstlisting}

%}}}

\section{Transparency}
%{{{

\begin{lstlisting}
# Pure red will be transparent:
convert -transparent red input.gif output.gif
\end{lstlisting}

%}}}

\section{Scale}
%{{{

Inline (while reading) multiple image processing:
\begin{lstlisting}
# Creates a 120x120 PNG thumbnail of each input image in the current directory. Each
# image is processed as follows: read -> resize -> write.
convert 'tempete*.png[120x120]' tempete_thumbnail%03d.png
animate tempete_thumbnail*.png
\end{lstlisting}

%}}}

\section{Crop}
%{{{

\begin{lstlisting}
identify tempete000.png
# tempete000.png PNG 352x288 352x288+0+0 8-bit DirectClass 200KB 0.000u 0:00.000
#                                    ^ ^
#     These offsets are important because should be added to "xo" and "yo"!

convert 'tempete000.png[100x200+50+50]' - | display -
#                       ^^^ ^^^ ^^ ^^
#                       col row xo yo
# col = column, row = row, xo = x offset, yo = y offset.

# The same as previous example, but using (inline, while reading) cropping instead resizing.
convert '*.jpg[120x120+10+5]' thumbnail%03d.png
\end{lstlisting}

%}}}

\section{Manipulating contrast}
%{{{

\begin{itemize}
\item The
  \href{http://en.wikipedia.org/wiki/Contrast_(vision)}{contrast} is
  the difference in luminance to make and object distinguishable.
\begin{lstlisting}
# Increasing:
convert -constrast input.png output.png

# Decreasing:
convert +constrast input.png output.png
\end{lstlisting}

\end{itemize}

%}}}

\section{Dithering}
%{{{

\begin{itemize}
\item Reduces the number of colors.
\begin{lstlisting}
# Using 2 colors:
convert -dither -monochrome input.png output.png

# Using 4 colors:
convert -dither -colors 4 input.png output.png
\end{lstlisting}

\end{itemize}

%}}}

\section{Equalizing}
%{{{

\begin{itemize}
\item Increases the dynamic range of the colors of the image.
\begin{lstlisting}
convert -equalize input.png output.png
\end{lstlisting}

\end{itemize}

%}}}

%\section{Tinting}
\section{Normalizing}
%{{{

\begin{itemize}
\item Increases the dynamic range of the colors of the image.
\begin{lstlisting}
convert -normalize input.png output.png
\end{lstlisting}

\end{itemize}

%}}}

%\section{Enhancing}
%\section{Modulating}
%\section{Flattening}

\section{Bluring}
%{{{

\begin{itemize}
\item Low pass filtering:
\begin{lstlisting}
convert -blur 10 input.png output.png
#             ^
#             |
#          radius

conver -blur 7x3 input.opng output.png
#             ^
#             |
#          size of the convolutional mask
\end{lstlisting}
\end{itemize}


%}}}

\section{The \emph{charcoal} effect}
%{{{

\begin{itemize}
\item High pass filtering:
\begin{lstlisting}
convert -charcoal 2 input.png output.png
#                 ^
#                 |
#       thicknes of the boundaries
\end{lstlisting}

\end{itemize}

%}}}

%\section{colorize}
\section{The \emph{impode} effect}
%{{{

\begin{itemize}
\item A sink effect:
\begin{lstlisting}
convert -implode 2 input.png output.png
#                ^
#                |
#             factor
\end{lstlisting}

\end{itemize}
%}}}

\section{Denoising}
%{{{

\begin{itemize}
\item Removes noise:
\begin{lstlisting}
convert -noise 3 input.png output.png
#              ^
#              |
#            radii
\end{lstlisting}

\end{itemize}

%}}}

\section{The \emph{paint} effect}
%{{{

\begin{itemize}
\item Simulates oil painting:
\begin{lstlisting}
convert -paint 3 input.png output.png
#              ^
#              |
#            radii
\end{lstlisting}

\end{itemize}
%}}}

\section{The \emph{radial-blur} effect}
%{{{

\begin{itemize}
\item Simulates oil painting:
\begin{lstlisting}
convert -radial-blur 30 input.png output.png
#                     ^
#                     |
#                   angle
\end{lstlisting}

\end{itemize}
%}}}

%\section{raise}

\section{Sementing}
%{{{

\begin{itemize}
\item Automatic image segmentation:
\begin{lstlisting}
convert -segment 1.75x2.0 input.png output.png
#                  ^   ^
#                  |   |
#                  | smoothing
#                clustering
\end{lstlisting}

\end{itemize}
%}}}

\section{The \emph{sepia-tone} effect}
%{{{

\begin{itemize}
\item A vintage effect:
\begin{lstlisting}
convert -sepia-tone 75% input.png output.png
#                    ^
#                    |
#                percentage
\end{lstlisting}

\end{itemize}
%}}}

\section{The \emph{shade} effect}
%{{{

\begin{itemize}
\item Produces shades:
\begin{lstlisting}
convert -shade 0x10 input.png output.png
#              ^  ^
#              |  |
#              | elevation of the source of light
#             azimuth
\end{lstlisting}

\end{itemize}
%}}}

\section{Sharpenning}
%{{{

\begin{itemize}
\item Using a Gaussian filtering to enhance the color boundaries:
\begin{lstlisting}
convert -sharpen 3x6 input.png output.png
#                ^  ^
#                |  |
#                | standard deviation
#               radii
\end{lstlisting}

\end{itemize}
%}}}

\section{Negating}
%{{{

\begin{itemize}
\item A negative effect:
\begin{lstlisting}
convert -negate input.png output.png
\end{lstlisting}

\end{itemize}
%}}}

\section{The \emph{solarize} effect}
%{{{

\begin{itemize}
\item Another negative effect:
\begin{lstlisting}
convert -solarize 30 input.png output.png
#                  ^
#                  |
#               factor
\end{lstlisting}

\end{itemize}

%}}}

\section{The \emph{spread} effect}
%{{{

\begin{itemize}
\item Swaps pixels randomly:
\begin{lstlisting}
convert -spread 7 input.png output.png
#               ^
#               |
#             radii
\end{lstlisting}
\end{itemize}

%}}}

%\section{swirl}
%\section{threshold}
%\section{unsharp}
\section{The \emph{wave} effect}
%{{{

\begin{itemize}
\item Shakes the image:
\begin{lstlisting}
convert -wave 10x90 input.png output.png
#              ^  ^
#              |  | 
#              | frequency
#           amplitude
\end{lstlisting}
\end{itemize}

%}}}

%\section{virtual-pixel}

%}}}

\chapter{Editing}
%{{{

\section{Annotating}
%{{{

\begin{itemize}
\item Put text inside images.
\begin{lstlisting}
convert -annotate 0x0+30+40 -stroke blue -fill red ``text'' input.png output.png
#                     ^  ^
#                     |  |
#                     | Y offset
#                    X offset      
\end{lstlisting}

\end{itemize}

%}}}

\section{Removing metadata}
%{{{

\begin{lstlisting}
identify -verbose input.jpg
convert -strip input.jpg output.jpg
identify -verbose output.jpg
\end{lstlisting}

%}}}

\section{Change metadata}
%{{{

\begin{lstlisting}
convert -comment ``This image was created by ...'' input.jpg output.jpg
convert -density 150x150 -units PixelsPerInch input.jpg output.jpg
\end{lstlisting}

%}}}

\section{Scaling}
%{{{

\begin{enumerate}

\item Absolute size:

\begin{lstlisting}
convert -resize 352x288 input.jpg output.jpg
\end{lstlisting}

\item Maximum size (not larger than) and conserving the width/height ratio:

\begin{lstlisting}
convert -resize 352x288< input.jpg output.jpg
\end{lstlisting}

\item Minimum size (not smaller than) and conserving the width/height ratio:

\begin{lstlisting}
convert -resize 352x288> input.jpg output.jpg
\end{lstlisting}

\item Percentage change:

\begin{lstlisting}
convert -resize 25% input.jpg output.jpg
\end{lstlisting}

\item Maximun number of pixels:

\begin{lstlisting}
convert -resize 25000@ input.jpg output.jpg
\end{lstlisting}

\item Sampling (changes both, the resolution and the image size):
\begin{lstlisting}
convert -sample 250% input.jpg output.jpg
\end{lstlisting}

\item Resampling (in dots per inch). It changes the resolution, but not the size:

\begin{lstlisting}
convert -resample 150x150 input.jpg output.jpg
\end{lstlisting}

\item Using a faster (but unaccurate) algorithm:

\begin{lstlisting}
convert -scale 200% input.jpg output.jpg
\end{lstlisting}

\item Thumbnails:

\begin{lstlisting}
# Only one image:
convert -thumbnail 10% input.jpg output.jpg
\end{lstlisting}

\begin{lstlisting}
# A group of images (this example does not overwrites the original images):
mogrify -format png -thumbnail 15% *.jpg
\end{lstlisting}


\end{enumerate}

%}}}

\section{Rotating}
%{{{

\begin{lstlisting}
convert -rotate -75 input.png output.png
\end{lstlisting}

%}}}

\section{Flipping}
%{{{

\begin{lstlisting}
# Vertically:
convert -flip input.png output.png

# Horizontaally:
convert -flop input.png output.png
\end{lstlisting}

%}}}

\section{Shearing}
%{{{

\begin{itemize}
\item Put the input image at an angle.

\begin{lstlisting}
convert -shear 45x10 input.png output.png
#               ^  ^
#               |  |
#               | vertical shearing
#             horizontal
\end{lstlisting}
\end{itemize}

%}}}

%\section{Rolling}
\section{Averaging}
%{{{

\begin{lstlisting}
convert -average *.png output.png
\end{lstlisting}

%}}}

\section{Cropping}
%{{{

\begin{enumerate}

\item Extract a subregion:

\begin{lstlisting}
convert -crop 999x768+100+200 input.png output.png
#              ^   ^   ^   ^
#              |   |   |   |
#              |   |   | Y offset
#              |   | X offset
#              | height of the extracted image
#            width of the extracted image
\end{lstlisting}

\item Remove a border:

\begin{lstlisting}
convert -shave 10x10 input.jpg output.jpg
\end{lstlisting}

\item Remove the constant border:

\begin{lstlisting}
convert -trim input.jpg output.jpg
\end{lstlisting}

\item Remove the similar border:

\begin{lstlisting}
convert -fuzz 10% input.jpg output.jpg
\end{lstlisting}

\end{enumerate}

%}}}

\section{Chopping}
%{{{

\begin{itemize}
\item Removes columns and rows (both, at the same time) of an image.
\begin{lstlisting}
convert -chop 500x600+700+800 input.png output.png
#              ^   ^   ^   ^
#              |   |   |   |
#              |   |   | Y offset
#              |   | X offset
#              | height of the removed region
#            width of the removed region
\end{lstlisting}

\end{itemize}

%}}}

\section{Adding images}
%{{{

\begin{lstlisting}
convert image1.jpg image2.png image3.tif output.pdf
\end{lstlisting}

%}}}

\section{Inserting images}
%{{{

\begin{lstlisting}
# Inserts ``image.png'' as the 3rd image:
convert -insert 3 image.png input.mng output.gif
\end{lstlisting}

%}}}

\section{Deleting images}
%{{{

\begin{lstlisting}
# Deletes the 3rd, 4th and 10th images:
convert -delete 3,4,10 input.mng output.gif
\end{lstlisting}

%}}}

\section{Swapping images}
%{{{

\begin{lstlisting}
# Swaps the 3rd, 4th images:
convert -delete 3,4 input.mng output.gif
\end{lstlisting}

%}}}

\section{Appending}
%{{{

\begin{lstlisting}
convert top.png bottom.png -append result.png
\end{lstlisting}

%}}}

%}}}
